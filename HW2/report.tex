\documentclass[12pt, a4paper]{article}

\usepackage{multicol}
\usepackage{geometry}
\usepackage{setspace}
\usepackage{CJKutf8}
\usepackage{amsmath}
\usepackage{listings}
\usepackage{graphicx}

\title{
    \textbf{Report Title} \\
    \large Report Subtitle \\
    \small Tracy Liu
    \author Tracy Liu
    \date{}
}

\geometry{a4paper,left=2cm,right=2cm,top=2.8cm,bottom=3.2cm}
\setlength{\columnsep}{1cm}
\setlength{\baselineskip}{100pt}
\linespread{1.2}
\graphicspath{ {./images/} }

\begin{document}
    \begin{CJK*}{UTF8}{bsmi}

    % title
    \begin{center}
        \LARGE\textbf{2019 Fall Introduction to Operating Systems} \\
        \large Homework 2 \\
        \small 0616015 劉姿利 \\
    \end{center}

    % \begin{multicols*}{2}

    % contents
    \section{Briefly describe about your data structure for recording process time or anything you need to record.}
        \subsection{struct P}
        \begin{lstlisting}[language=C++]
struct P {
    int id;
    int prior;
    int arrival;
    int burst;
    int last;
    bool operator>(const P & rhs) const {
        return (prior > rhs.prior) ||
               (prior == rhs.prior && last > rhs.last) ||
               (prior == rhs.prior && last == rhs.last && id > rhs.id);
    }
};
        \end{lstlisting}
          用來儲存\ process\ 資料。\\
          因為想要使用一些需要排序的結構,所以要將\ operator overloading\ 寫好。\\

        \subsection{FCFS}
              我使用\ C++ STL\ 中的\ vector\ 來儲存\ process informations,接著以\ process\ 的\ arrival time\ 為基準對\ vector\ 做\ sorting,最後紀錄一些重要時間點\ timer\ 來計算\ waiting time\ 以及\ turnaround time。
            
        \subsection{Priority Scheduling}
              至於\ Priority Scheduling ,我用了\ C++ STL priority\_queue,它有內建\ heap\ 結構,將\ operator overloading\ 寫好就可以每次取出\ priority\ 最高的\ process。

        \subsection{Round-Robin}
            Round-Robin\ 的話我使用\ C++ STL\ 中的\ queue,它有\ first in first out\ 的特性,我就可以按照\ arrival time\ 將\ process\ 取出。

        \subsection{Multilevel Feedback Queue}
              這次\ multilevel feedback queue\ 要將\ RR\ 跟\ priority scheduling\ 組合在一起,所以我會先宣告一個\ queue\ 來處理\ RR\ 的步驟,再宣告一個\ priority\_queue\ 來處理\ priority scheduling,在\ queue\ 處理完成的\ process\ 就會被\ push\ 進入\ priority queue。

    \section{Some problems you meet and how to resolve.}
        \subsection{Problem}
          一開始不知道如何計算\ waiting time。
        \subsection{Solution}
          後來想到可以在放入\ queue\ 或\ priority\_queue\ 之前記錄時間,存在\ struct P\ 中一個叫\ last\ 的變數裡面,並在下次從\ queue / priority\_queue\ 取出時拿當前時間相減,再加到那個\ process\ 的\ waiting time\ 上就可以了。

    \section{What you learned from doing OS hw2 and something you want to discuss with TAs.}
          為了保證寫出來的\ simulation\ 是對的,這次寫了一個生測資的東西,助教的\ email\ 中有寫到測資保證在跑\ Q1\ 的時候\ Q0\ 不會有東西進來,我花了一段時間才想到要如何\ random\ 用\ C++\ 生出這樣的測資,覺得挺有趣的。
        


    % \end{multicols*}
    \end{CJK*}
\end{document}
